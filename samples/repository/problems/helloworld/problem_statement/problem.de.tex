% flavour text, presentation of the problem, may contain images and quotes
% image:
%\begin{figure}[h]
%\center\includegraphics[width=0.3\textwidth]{name.png}
%\end{figure}

% quotation:
%\begin{myquote}
%bla bla 
%\end{myquote}
\problemname{Hello World!}

Dies ist vermutlich das erste Problem, das dir gestellt wird. Es soll dir helfen, deine Systeme einzurichten und zu testen. Löse dieses Problem als erstes um sicherzustellen, dass du später nicht von technischen Schwierigkeiten überrascht wirst.

Wir möchten dir gerne Lea vorstellen. Du wirst sie in vielen Problemen hier antreffen. Wenn du einmal alle gelesen hast, wirst du sie wirklich gut kennen.

Lea ist ein freundlicher Mensch, der gerne jeden mit ``Hello'' begrüßt (Lea ist auch sehr international). Aber sie möchte auch nicht zu jeder Person, die sie trifft, das Gleiche sagen. Daher verwendet sie ein einfaches Muster: Um Bob zu grüßen, sagt sie ``Hello Bob!'', während es beispielsweise passend erscheint, Peter mit ``Hello Peter!'' anzureden. Hilf ihr und schlage ihr vor, welche Phrase sie jeweils verwenden kann.

\section*{Eingabe}

Die erste Zeile der Eingabe enthält eine Ganzzahl $t$ ($1 \leq t \leq 20$). Darauf folgen $t$ Testfälle.

Jeder Testfall besteht aus einer Zeile mit einem Namen $s$. Diese Zeichenkette besteht stets aus mindestens einem und höchstens $100$ Groß- oder Kleinbuchstaben.

\section*{Ausgabe}

Gib für jeden Testfall eine Zeile ``Case \#$i$: Hello $s$!'' aus, wobei $i$ bei $1$ beginnend die Nummer des Testfalls ist. Jede Zeile der Ausgabe soll mit einem Zeilenumbruch enden.

